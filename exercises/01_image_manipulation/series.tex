\documentclass[a4paper]{scrreprt}

% Uncomment to optimize for double-sided printing.
% \KOMAoptions{twoside}

% Set binding correction manually, if known.
% \KOMAoptions{BCOR=2cm}

% Localization options
\usepackage[english]{babel}
\usepackage[T1]{fontenc}
\usepackage[utf8]{inputenc}

% Quotations
\usepackage{dirtytalk}

% Floats
\usepackage{float}

\usepackage{numbertabbing}

% Enhanced verbatim sections. We're mainly interested in
% \verbatiminput though.
\usepackage{verbatim}

% Automatically remove leading whitespace in lstlisting
\usepackage{lstautogobble}

% PDF-compatible landscape mode.
% Makes PDF viewers show the page rotated by 90°.
\usepackage{pdflscape}

% Advanced tables
\usepackage{array}
\usepackage{tabularx}
\usepackage{longtable}

% Fancy tablerules
\usepackage{booktabs}

% Graphics
\usepackage{graphicx}

% Current time
\usepackage[useregional=numeric]{datetime2}

% Float barriers.
% Automatically add a FloatBarrier to each \section
\usepackage[section]{placeins}

% Custom header and footer
\usepackage{fancyhdr}

\usepackage{geometry}
\usepackage{layout}

% Math tools
\usepackage{mathtools}
% Math symbols
\usepackage{amsmath,amsfonts,amssymb}
\usepackage{amsthm}

\DeclarePairedDelimiter\ceil{\lceil}{\rceil}
\DeclarePairedDelimiter\floor{\lfloor}{\rfloor}

% General symbols
\usepackage{stmaryrd}

\DeclarePairedDelimiter\abs{\lvert}{\rvert}

% Indistinguishable operator (three stacked tildes)
\newcommand*{\diffeo}{% 
  \mathrel{\vcenter{\offinterlineskip
  \hbox{$\sim$}\vskip-.35ex\hbox{$\sim$}\vskip-.35ex\hbox{$\sim$}}}}

% Bullet point
\newcommand{\tabitem}{~~\llap{\textbullet}~~}

\floatstyle{ruled}
\newfloat{algo}{htbp}{algo}
\floatname{algo}{Algorithm}
% For use in algorithms
\newcommand{\str}[1]{\textsc{#1}}
\newcommand{\var}[1]{\textit{#1}}
\newcommand{\op}[1]{\textsl{#1}}

\pagestyle{plain}
% \fancyhf{}
% \lhead{}
% \lfoot{}
% \rfoot{}
% 
% Source code & highlighting
\usepackage{listings}

% SI units
\usepackage[binary-units=true]{siunitx}
\DeclareSIUnit\cycles{cycles}

% Convenience commands
\newcommand{\mailsubject}{33107 - Document Image Analysis - Series 1}
\newcommand{\maillink}[1]{\href{mailto:#1?subject=\mailsubject}
                               {#1}}

% Should use this command wherever the print date is mentioned.
\newcommand{\printdate}{\today}

\subject{33107 - Document Image Analysis}
\title{Series 1}

\author{Michael Senn \maillink{michael.senn@students.unibe.ch} - 16-126-880}

\date{\printdate}

% Needs to be the last command in the preamble, for one reason or
% another. 
\usepackage{hyperref}

\begin{document}
\maketitle


\setcounter{chapter}{0}

\chapter{Series 1}

\section{GitHub repository}

The code can be found on \url{https://github.com/Lavode/33107-document-image-analysis}

\section{Resizing algorithm}

For resizing, nearest-neighbour interpolation was implemented. A pixel $(x, y)$
in the output image is set the `nearest' corresponding pixel in the input
image.  For upscaling this means that multiple output pixels will be equal to
one input pixel, for downscaling this means that output pixels will be an
evenly spaced subset of input pixels.

\begin{algo}
  \vbox{
    \small
    \begin{numbertabbing}
      xxxx\=xxxx\=xxxx\=xxxx\=xxxx\=xxxx\=MMMMMMMMMMMMMMMMMMM\=\kill
	  \textbf{Input} \\
	  \> \var{I}: Input image \\
	  \> \var{f}: Scale factor. f > 1 for upscaling, f < 1 for downscaling \\
	  \> \var{(x, y)}: Output coordinates \\
	  \textbf{Algorithm} \\
      \> \var{x'} := $\floor{x * f}$ \\
      \> \var{y'} := $\floor{y * f}$ \\
	  \> \textbf{Return} I[x', y']
    \end{numbertabbing}
  }
  \caption{Nearest-neighbour interpolation} 
  \label{alg:nearest_neighbour}
\end{algo}

\section{Convolution algorithm}

let $K$ be the $n \times n$ convolution kernel. Let $I$ be the $a \times b$ input
image.

\subsection{2D edge padding}

In a first step, the input image is padded. $pad := \floor{n / 2}$ pixels of
padding are added on each side, allowing the image-kernel multiplication to
happen for all pixels. Padding is done as edge-padding, where the value of a
padded pixel $(x, y)$ is determined by the pixel of the input image $(x', y')$
which is `closest' - that is the pixel with the lowest Manhattan distance.

This can be trivially implemented by first copying the leftmost and rightmost
columns of the input image an appropriate amount of times on the left
respectively right, and then copying the topmost and bottommost rows of the
padded image on the top respectively bottom.

\subsection{Convolution}

The convolution kernel is first flipped in both the horizontal and vertical
direction, to accomodate for the definition of convolution iterating over the
elements of the image, and the elements of the kernel, in opposite directions.
This way convolution can be implemented as simple pairwise multiplication. This
has no effect on symmetrical kernels.

Convolution then takes place by iterating over each pixel $(x, y)$ of the input
image - excluding the pixels which were added as padding - and performing a
component-wise matrix multiplication + sum of the convolution kernel $K$ with
the $n \times n$ window of pixels surrounding $(x, y)$.

\subsection{Blurring}

For blurring, a $5 \times 5$ gaussian kernel from the lecture notes was used:
\[
		K = \begin{pmatrix}
				1 & 3 & 5 & 3 & 1 \\
				3 & 14 & 23 & 14 & 3 \\
				5 & 23 & 38 & 23 & 5 \\
				3 & 14 & 23 & 14 & 3 \\
				1 & 3 & 5 & 3 & 1
		\end{pmatrix} / 234
\]

It works by looking at the $5 \times 5$ pixels surrounding each pixel $(x, y)$,
and setting the corresponding pixel of the output as the average of these
pixels, weighted with the corresponding factors from the kernel.

The convolution $K \cdot I$ was used directly as output.

\subsection{Edge detection}

For edge detection, two Sobel kernels were used:
\begin{align*}
		K_x & = \begin{pmatrix}
				1 & 0 & -1 \\
				2 & 0 & -2 \\
				1 & 0 & -1
		\end{pmatrix} \\
		K_y &= \begin{pmatrix}
				1 & 2 & 1 \\
				0 & 0 & 0 \\
				-1 & -2 & -1
		\end{pmatrix}
\end{align*}

The gradient was calculated as the convolution of the input image with $K_x$
respectively $K_y$:
\[
		\nabla I = \begin{pmatrix}
				K_x \cdot I \\
				K_y \cdot I
		\end{pmatrix}
\]

And finally its magnitude:
\[
		\abs{\nabla I} = \sqrt{(K_x \cdot I)^2 + (K_y \cdot I)^2}
\]

Which was used as the output image.

The Sobel kernel combines a smoothing operation - to remove noise - with an
approximation of the gradient in .

\end{document}
